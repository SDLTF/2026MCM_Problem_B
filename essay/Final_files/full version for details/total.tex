%% Options for packages loaded elsewhere
\PassOptionsToPackage{unicode}{hyperref}
\PassOptionsToPackage{hyphens}{url}

\documentclass[12pt]{article}
\setcounter{secnumdepth}{-\maxdimen} % remove section numbering
\usepackage{iftex}
\usepackage{graphicx}
\makeatletter
\newsavebox\pandoc@box
\newcommand*\pandocbounded[1]{% scales image to fit in text height/width
	\sbox\pandoc@box{#1}%
	\Gscale@div\@tempa{\textheight}{\dimexpr\ht\pandoc@box+\dp\pandoc@box\relax}%
	\Gscale@div\@tempb{\linewidth}{\wd\pandoc@box}%
	\ifdim\@tempb\p@<\@tempa\p@\let\@tempa\@tempb\fi% select the smaller of both
	\ifdim\@tempa\p@<\p@\scalebox{\@tempa}{\usebox\pandoc@box}%
	\else\usebox{\pandoc@box}%
	\fi%
}
\usepackage[UTF8]{ctex}
\usepackage{amsmath, amssymb, amsthm}
\usepackage{graphicx}
\usepackage{hyperref}
\usepackage{algorithm}
\usepackage{xcolor}
\usepackage{cite}
\usepackage{setspace}
\usepackage[margin=1in]{geometry}
\usepackage{booktabs}
\usepackage{longtable}
\usepackage{multirow}
\usepackage{caption}
\usepackage{subcaption}
\usepackage{float}
\usepackage{tcolorbox}
\usepackage{tikz}
\usepackage{pgfplots}
\pgfplotsset{compat=1.17}
\usepackage{algpseudocode}
\usepackage{appendix}
\usepackage{minted}
\ifPDFTeX
\usepackage[T1]{fontenc}
\usepackage[utf8]{inputenc}
\usepackage{textcomp} % provide euro and other symbols
\else % if luatex or xetex
\usepackage{unicode-math} % this also loads fontspec
\defaultfontfeatures{Scale=MatchLowercase}
\defaultfontfeatures[\rmfamily]{Ligatures=TeX,Scale=1}
\fi
\usepackage{lmodern}
\ifPDFTeX\else
% xetex/luatex font selection
\fi
% Use upquote if available, for straight quotes in verbatim environments
\IfFileExists{upquote.sty}{\usepackage{upquote}}{}
\IfFileExists{microtype.sty}{% use microtype if available
	\usepackage[]{microtype}
	\UseMicrotypeSet[protrusion]{basicmath} % disable protrusion for tt fonts
}{}
\makeatletter
\@ifundefined{KOMAClassName}{% if non-KOMA class
	\IfFileExists{parskip.sty}{%
		\usepackage{parskip}
	}{% else
		\setlength{\parindent}{0pt}
		\setlength{\parskip}{6pt plus 2pt minus 1pt}}
}{% if KOMA class
	\KOMAoptions{parskip=half}}
\makeatother
\setlength{\emergencystretch}{3em} % prevent overfull lines
\providecommand{\tightlist}{%
	\setlength{\itemsep}{0pt}\setlength{\parskip}{0pt}}
\usepackage{bookmark}
\IfFileExists{xurl.sty}{\usepackage{xurl}}{} % add URL line breaks if available
\urlstyle{same}
\hypersetup{
	hidelinks,
	pdfcreator={LaTeX via pandoc}}
\title{\textbf{高等代数(I)知识点 \\ All in ONE\\ v 4.0}}
\author{SDLTF 四川大学}
\date{}

\begin{document}

\section{3. 数据处理与参数估计 (Data Processing and Parameter
Estimation)}\label{3-ux6570ux636eux5904ux7406ux4e0eux53c2ux6570ux4f30ux8ba1-data-processing-and-parameter-estimation}

为了确保物流网络优化模型的科学性,本章基于历史航天数据和物理环境预测,对影响地月物流系统的核心动态参数进行定量校准。我们主要关注火箭技术的成熟度趋势、规模经济下的成本演化以及太空电梯在复杂空间环境下的退化特征。

\subsection{3.1 火箭效能:基于 Logistic
回归的技术成熟度模型}\label{31-ux706bux7badux6548ux80fdux57faux4e8e-logistic-ux56deux5f52ux7684ux6280ux672fux6210ux719fux5ea6ux6a21ux578b}

火箭系统的发射频次(Cadence)受限于发射场周转效率、载具复用速度及技术熟练度。

\subsubsection{3.1.1
数据来源与处理}\label{311-ux6570ux636eux6765ux6e90ux4e0eux5904ux7406}

我们采集了 SpaceX 在 2012 年至 2024
年间的真实发射数据作为样本。这些数据涵盖了从 Falcon 9 初期到 Starship
试验期的演化过程。通过对异常年份(如早期技术瓶颈期)进行平滑处理,我们提取了描述运力增长的时间序列。

\subsubsection{3.1.2
逻辑回归拟合}\label{312-ux903bux8f91ux56deux5f52ux62dfux5408}

考虑到物理基础设施(如发射工位、推进剂加注速度)存在上限,我们采用
\textbf{Logistic S型增长模型} 描述单发射场的年运力
\(\beta_{cadence}(t)\):

\[\beta_{cadence}(t) = \frac{L}{1 + e^{-k(t - t_0)}}\]

\begin{itemize}
\item
  \textbf{饱和上限 (\(L\))}:根据 Starship
  的设计参数及地面支持系统的物理极限,设定单个发射场的年发射能力上限为
  500 次。
\item
  \textbf{拟合结果}:如图 1 所示,拟合曲线显示到 2050
  年,单一发射场的年吞吐量将稳定在 \(400 \sim 450\)
  次,标志着火箭技术进入高度成熟的平台期\pandocbounded{\includegraphics[keepaspectratio,alt={}]{C:/Users/31529/Desktop/MCMworkspace/pictures/parameter_estimation_beta_fixed_v4.png}}
\end{itemize}

\subsection{3.2 经济平衡:莱特定律与环境税 (Economic
Equilibrium)}\label{32-ux7ecfux6d4eux5e73ux8861ux83b1ux7279ux5b9aux5f8bux4e0eux73afux5883ux7a0e-economic-equilibrium}

\subsubsection{3.2.1 学习曲线与边际成本 (Wright's
Law)}\label{321-ux5b66ux4e60ux66f2ux7ebfux4e0eux8fb9ux9645ux6210ux672c-wrights-law}

基于航天工业的规模效应,我们引入\textbf{莱特定律}来修正火箭发射的单位成本
\(C_u\):

\[C_u(q) = C_1 \cdot q^{-b}, \quad b = -\log_2(f)\]

其中 \(q\) 是累计运输量,\(f\) 为学习率(取
0.85)。这意味着随着任务向一亿吨迈进,单位成本将通过经验积累实现非线性衰减。

\subsubsection{3.2.2 非线性环境惩罚模型 (Environmental Externality
Model)}\label{322-ux975eux7ebfux6027ux73afux5883ux60e9ux7f5aux6a21ux578b-environmental-externality-model}

为了量化密集发射对平流层和近地轨道的破坏,我们定义了环境惩罚函数
\(E(f)\):

\[E(f) = \eta \cdot \left( \frac{f}{f_{crit}} \right)^p, \quad p \approx 2.0\]

这里 \(f_{crit}\) 为大气自净频率阈值。当发射频率 \(f\)
超过临界值时,环境修复成本将呈平方级增长。下面这个图片展示了环境因素带来的影响。

\pandocbounded{\includegraphics[keepaspectratio,alt={}]{C:/Users/31529/Desktop/MCMworkspace/pictures/cost_dynamics_intersection.png}}

\subsection{3.3
太空电梯退化:环境恶化反馈回路}\label{33-ux592aux7a7aux7535ux68afux9000ux5316ux73afux5883ux6076ux5316ux53cdux9988ux56deux8def}

与火箭技术的上升趋势相反,太空电梯作为长达 100,000 km
的静态目标,面临着严峻的空间环境威胁。

\subsubsection{3.3.1 凯斯勒效应 (Kessler Syndrome)
的影响}\label{331-ux51efux65afux52d2ux6548ux5e94-kessler-syndrome-ux7684ux5f71ux54cd}

根据欧洲空间局 (ESA)
的预测,低地轨道碎片的密度将持续增加。我们建立了可用性因子
\(\alpha(t)\),用以描述电梯在一年中能够正常运行的比例。

\subsubsection{3.3.2
效率衰减方程}\label{332-ux6548ux7387ux8870ux51cfux65b9ux7a0b}

\(\alpha(t)\) 受限于撞击频率与自动化维修速度的博弈:

\[\alpha(t) = 1 - \frac{\lambda_0 (1+r_d)^{t-t_0} \times \tau_0 (1-r_s)^{t-t_0}}{365}\]

\begin{itemize}
\item
  \textbf{参数标定}:其中碎片增长率 \(r_d\) 设定为 1.5\%,维修技术进步率
  \(r_s\) 设定为 0.5\%。
\item
  \textbf{趋势背离分析}:如图 2
  所示,红线代表的电梯效率由于环境退化而缓慢下降,而绿线代表的火箭效能持续上升。这种\textbf{趋势背离
  (Divergent Trends)}
  构成了我们``混合运输策略''的核心论点:在项目后期,必须通过增加火箭份额来对冲电梯的停机风险。
\end{itemize}

\pandocbounded{\includegraphics[keepaspectratio,alt={}]{C:/Users/31529/Desktop/MCMworkspace/pictures/alpha_beta_divergence.png}}

\subsection{3.4
可靠性与风险参数标定}\label{34-ux53efux9760ux6027ux4e0eux98ceux9669ux53c2ux6570ux6807ux5b9a}

为了在后续章节中进行蒙特卡洛压力测试,我们基于历史数据设定了系统的风险分布:

\begin{itemize}
\item
  \textbf{载具可靠性}:到 2050 年,任务成功率设定为 99.98\%。
\item
  \textbf{干扰因素}:引入遵循泊松分布的随机停机事件(如天气、碎片规避),平均每年的计划外停机时间基准设定为
  14 天。
\end{itemize}

\section{4. 综合物流模型:协同动态调度框架 (The Integrated Logistics
Model: A Collaborative Dynamic
Framework)}\label{4-ux7efcux5408ux7269ux6d41ux6a21ux578bux534fux540cux52a8ux6001ux8c03ux5ea6ux6846ux67b6-the-integrated-logistics-model-a-collaborative-dynamic-framework}

本章构建了一个动态混合物流模型(ILM),旨在解决 2050
年后地月运输中\textbf{规模(Scale)、时效(Timing)与鲁棒性(Resilience)}之间的深层矛盾。

\subsection{4.1 双模态博弈与协同逻辑 (Dual-Mode Synergy \& Game
Logic)}\label{41-ux53ccux6a21ux6001ux535aux5f08ux4e0eux534fux540cux903bux8f91-dual-mode-synergy--game-logic}

我们拒绝将太空电梯与火箭视为简单的运力相加,而是将其建模为一个基于排队论的\textbf{``基准-补偿''协同系统}:

\begin{itemize}
\item
  \textbf{太空电梯 (The Steady-State
  Anchor)}:作为系统的``底色'',其核心价值在于极高的能源效率和零碳排放。它承担了
  24/7 的标准化物资运输。然而,其物理瓶颈在于无法处理瞬时浪涌载荷。
\item
  \textbf{火箭舰队 (The Agile Surge
  Capacity)}:火箭是系统的\textbf{``安全阀''和``带宽放大器''}。通过在特定的轨道窗口期释放瞬时能量,火箭舰队抵消了由于天体力学约束导致的物流堆积,并提供了必要的\textbf{浪涌容量
  (Surge Capacity)} 以确保工期不被延误。
\end{itemize}

\subsection{4.2 数学模型:离散脉冲与连续流的耦合 (Mathematical
Formulation)}\label{42-ux6570ux5b66ux6a21ux578bux79bbux6563ux8109ux51b2ux4e0eux8fdeux7eedux6d41ux7684ux8026ux5408-mathematical-formulation}

系统的总运力流 \(M_{total}\) 被定义为随时间 \(t\)
变化的非线性脉冲积分函数:

\[M_{total} = \int_{2050}^{2050+T} \left[ \Phi_E \cdot \alpha(t) + \sum_{k=1}^{n} \delta(t - k\tau) \cdot Q_{surge}(N) \right] dt\]

\subsubsection{4.2.1
关键项解析}\label{421-ux5173ux952eux9879ux89e3ux6790}

\begin{itemize}
\item
  \textbf{\(\Phi_E \cdot \alpha(t)\) (电梯连续产出)}:\(\Phi_E\)
  为设计运力(约 53.7 万吨/年),\(\alpha(t)\)
  为受环境退化(如轨道碎片密度 \(D(t)\))反馈调节的可用性因子。
\item
  \textbf{\(\delta(t - k\tau) \cdot Q_{surge}(N)\) (火箭脉冲产出)}:

  \begin{itemize}
  \item
    \textbf{\(\delta(\cdot)\)} 为狄拉克脉冲函数,代表每
    \(\tau \approx 28.5\) 天开启一次的离散发射窗口。
  \item
    \textbf{\(Q_{surge}(N)\)} 为 \(N\)
    个发射场在窗口期内能提供的最大瞬时吞吐量。
  \end{itemize}
\end{itemize}

\subsection{4.3 核心亮点:基于物理瓶颈的硬约束分析 (Constraint \&
Bottleneck
Analysis)}\label{43-ux6838ux5fc3ux4eaeux70b9ux57faux4e8eux7269ux7406ux74f6ux9888ux7684ux786cux7ea6ux675fux5206ux6790-constraint--bottleneck-analysis}

本模型通过引入以下``硬约束'',显著提升了工程说服力:

\subsubsection{4.3.1 轨道力学窗口 (TLI Windows)
与排队压力的离散化}\label{431-ux8f68ux9053ux529bux5b66ux7a97ux53e3-tli-windows-ux4e0eux6392ux961fux538bux529bux7684ux79bbux6563ux5316}

火箭发射受地月转移轨道窗口限制。我们的模型强制要求系统在每个窗口(约 28
天一次)内具备处理月度 30\% 以上物资的峰值能力。

\begin{itemize}
\item
  \textbf{论证结论}:如图 4.1 所示,这种\textbf{峰值-平均比 (PAR)}
  的极端性直接导出了\textbf{必须建设 25 个发射场}的基建建议。若
  \(N < 18\),系统将由于``窗口期拥堵''导致物资在 LEO
  轨道发生不可逆的积压。
\end{itemize}

\subsubsection{4.3.2 节点存储与缓冲动力学 (Buffering
Dynamics)}\label{432-ux8282ux70b9ux5b58ux50a8ux4e0eux7f13ux51b2ux52a8ux529bux5b66-buffering-dynamics}

我们引入缓冲水位方程 \(\frac{dB(t)}{dt} = \Phi_{in}(t) - \Phi_{out}(t)\)
来模拟 L1 点中转站的存量变化。

\begin{itemize}
\item
  \textbf{仿真发现}:25
  个发射场不仅是为了运输,更是为了确保在非窗口期,缓冲水位 \(B(t)\)
  不会跌破维持月球基地运行的\textbf{生存红线}。
\end{itemize}

\subsubsection{4.3.3 载荷耐受性分类 (Payload
Tiering)}\label{433-ux8f7dux8377ux8010ux53d7ux6027ux5206ux7c7b-payload-tiering}

模型将物资分为:

\begin{itemize}
\item
  \textbf{Tier 1 (敏感型)}:精密仪器,100\% 分配给加速度小于 \(0.1g\)
  的电梯。
\item
  \textbf{Tier 2 (标准型)}:建材与燃料,根据成本最优原则(见 3.2
  节莱特定律)在双模间动态切换。
\end{itemize}

\subsection{4.4 仿真结果:脉冲物流的可视化与``交叉点''分析 (Evolutionary
Trends)}\label{44-ux4effux771fux7ed3ux679cux8109ux51b2ux7269ux6d41ux7684ux53efux89c6ux5316ux4e0eux4ea4ux53c9ux70b9ux5206ux6790-evolutionary-trends}

\pandocbounded{\includegraphics[keepaspectratio,alt={}]{C:/Users/31529/Desktop/MCMworkspace/pictures/adaptive_allocation_trend.png}}

\textbf{分析与洞察}:

\begin{itemize}
\item
  \textbf{初期 (Phase I)}:电梯利用率维持在 95\%
  以上,作为绝对主力以降低碳排放。
\item
  \textbf{中期 (Phase II)}:随着火箭可靠性 \(\beta(t)\)
  提升且成本受学习曲线驱动下降,火箭开始承担 60\% 以上的``增量需求''。
\item
  \textbf{韧性验证}:如图 4
  所示,当模型模拟电梯由于轨道碎片撞击发生停机时,25
  个发射场形成的\textbf{浪涌容量}(由阶跃函数 \(H\) 描述,见第 7
  章)能够在 82 天内补齐物资缺口,保证了月球文明的生存韧性。
\end{itemize}

\section{5. 环境与经济优化:寻找地月物流的``黄金平衡点'' (Environmental
\& Economic Optimization: Finding the Global
Optimum)}\label{5-ux73afux5883ux4e0eux7ecfux6d4eux4f18ux5316ux5bfbux627eux5730ux6708ux7269ux6d41ux7684ux9ec4ux91d1ux5e73ux8861ux70b9-environmental--economic-optimization-finding-the-global-optimum}

本章旨在通过定量决策模型,解决地月物流规划中最核心的博弈:如何在有限的时间框架内完成一亿吨物资的运输,同时确保经济成本、环境足迹与系统韧性之间的动态平衡。

\subsection{5.1 优化指标与多目标损失函数 (Optimization Objectives and
Metrics)}\label{51-ux4f18ux5316ux6307ux6807ux4e0eux591aux76eeux6807ux635fux5931ux51fdux6570-optimization-objectives-and-metrics}

我们定义了一个综合压力指数(Global System Stress Index,
GSSI),作为评价物流策略优劣的核心判据。该指数由以下三个维度的非线性加权组成:

\begin{enumerate}
\def\labelenumi{\arabic{enumi}.}
\item
  \textbf{经济压力 (\(P_{eco}\))}:包含基建的资本支出
  (CAPEX)、长期运营支出 (OPEX) 以及受莱特定律驱动的边际成本演化。
\item
  \textbf{环境载荷
  (\(P_{env}\))}:量化火箭高频发射对地球大气层的碳排放影响及平流层扰动。
\item
  \textbf{基建疲劳与风险指数
  (\(P_{risk}\))}:量化由于轨道碎片积累导致的电梯完整性风险,以及由于发射窗口拥堵产生的物流积压风险。
\end{enumerate}

\subsection{5.2 真实 L-曲线分析:速度的代价 (The True L-Curve: The
Penalty of
Speed)}\label{52-ux771fux5b9e-l-ux66f2ux7ebfux5206ux6790ux901fux5ea6ux7684ux4ee3ux4ef7-the-true-l-curve-the-penalty-of-speed}

为了探索工期对环境和成本的影响,我们对 20 至 100
年的多种工期方案进行了仿真。

\pandocbounded{\includegraphics[keepaspectratio,alt={}]{C:/Users/31529/Desktop/MCMworkspace/pictures/true_l_curve.png}}

\pandocbounded{\includegraphics[keepaspectratio,alt={}]{C:/Users/31529/Desktop/MCMworkspace/pictures/cost_env_optimization_v2_fixed_v4.png}}

\textbf{分析与洞察}:

\begin{itemize}
\item
  \textbf{``不可逾越''的左侧高压区 (20-40年)}:如图所示,当工期压缩至 40
  年以下时,成本和环境压力曲线呈现出近乎垂直的爆发式增长。

  \begin{itemize}
  \item
    \textbf{物理原因}:极短工期迫使系统在缺乏``学习曲线''红利的前提下进行超大规模基建。同时,由于地球大气层的自净能力有限,密集的火箭发射会导致环境负外部性成本呈指数级上升。
  \item
    \textbf{物流瓶颈}:此时,即使部署 25
    个发射场,也无法在有限的地月转移(TLI)窗口内完成物理吞吐,导致系统因``脉冲式拥堵''而丧失功能。
  \end{itemize}
\item
  \textbf{帕累托前沿的出现}:随着工期向 60
  年移动,环境足迹与成本支出进入了收益最显著的优化区间,这构成了我们决策的底座。
\end{itemize}

\subsection{5.3 长期退化约束:为什么不能无限延期? (Infrastructure
Fatigue: The Limit of
Duration)}\label{53-ux957fux671fux9000ux5316ux7ea6ux675fux4e3aux4ec0ux4e48ux4e0dux80fdux65e0ux9650ux5ef6ux671f-infrastructure-fatigue-the-limit-of-duration}

如果说``快''受限于物理吞吐和环境瓶颈,那么``慢''则受限于\textbf{基建的生命周期}。

\pandocbounded{\includegraphics[keepaspectratio,alt={}]{C:/Users/31529/Desktop/MCMworkspace/pictures/global_optimum_analysis_fixed_v4.png}}

\textbf{分析与洞察}:

\begin{itemize}
\item
  \textbf{``右侧上升''的风险盆地 (70-100年)}:模型显示,当工期超过 70
  年时,综合压力指数(GSSI)开始反弹。

  \begin{itemize}
  \item
    \textbf{逻辑论证}:这并非由于运力不足,而是由于太空电梯在轨道碎片环境下的\textbf{累积失效概率}超过了阈值。长期的维护支出(OPEX)将逐步抵消规模效应带来的收益,且过长的周期会推迟月球自给自足(ISRU)红利的释放。
  \end{itemize}
\item
  \textbf{工程现实感}:我们的模型考虑了基建疲劳度,证明了物流系统存在一个``生存窗口'',在此窗口外,系统的复杂性风险将变得不可控。
\end{itemize}

\subsection{5.4 ``1-2-6 战略''的确立 (The Convergence: The "1-2-6"
Strategy)}\label{54-1-2-6-ux6218ux7565ux7684ux786eux7acb-the-convergence-the-1-2-6-strategy}

通过对 GSSI
盆地的最小值搜索,模型最终收敛于一个具有极高鲁棒性的全局最优解点,我们称之为
\textbf{``1-2-6 战略''}:

\begin{itemize}
\item
  \textbf{1 亿吨 (100M Tons)}:锁定核心任务目标。
\item
  \textbf{25 个发射场 (25 Launchpads)}:这是基于 TLI
  轨道窗口限制和``电梯突发故障''冗余计算出的\textbf{最低基建刚需}。这 25
  个场站提供的``浪涌容量''是确保物流链在灾难发生时不中断的``生命保险''。
\item
  \textbf{60 年工期 (60-Year
  Duration)}:这是成本、碳排放与基建风险三者叠加后的\textbf{帕累托最优解}。
\end{itemize}

\textbf{结论}:在 60 年框架下,系统实现了约 \textbf{29.2\%
的环境足迹减量},并将总预算控制在 \textbf{\$17.26 Trillion}
的可负担区间。如图 5 所示,这一解点位于压力盆地的最深处,即使在 ISRU
技术波动 \(\pm 10\%\) 的情况下,该解依然保持稳定,体现了卓越的决策弹性。

\section{6. 模型精修:自适应演化与离散窗口优化 (Model
Refinement)}\label{6-ux6a21ux578bux7cbeux4feeux81eaux9002ux5e94ux6f14ux5316ux4e0eux79bbux6563ux7a97ux53e3ux4f18ux5316-model-refinement}

为了提升模型的工程真实感,我们对基础模型进行了精修。本章重点引入了天体力学中的\textbf{离散发射窗口约束},并建立了相应的自适应反馈机制。

\subsection{6.1 自适应脉冲物流模型:宏观趋势与微观窗口的耦合 (Adaptive
Pulsed Logistics
Model)}\label{61-ux81eaux9002ux5e94ux8109ux51b2ux7269ux6d41ux6a21ux578bux5b8fux89c2ux8d8bux52bfux4e0eux5faeux89c2ux7a97ux53e3ux7684ux8026ux5408-adaptive-pulsed-logistics-model}

在长达 60
年的工期中,最优物流策略并非平滑流动的过程,而是在天体力学约束下的\textbf{离散脉冲博弈}。本节对基础模型进行精修,引入了基于轨道周期的自适应调度逻辑。

\subsubsection{6.1.1 离散发射窗口的数学描述 (Mathematical Discrete
Windows)}\label{611-ux79bbux6563ux53d1ux5c04ux7a97ux53e3ux7684ux6570ux5b66ux63cfux8ff0-mathematical-discrete-windows}

火箭运输受限于地月转移轨道(TLI)的相位窗口。我们将火箭的年度平均产出
\(\Phi_R(t)\) 修正为基于\textbf{狄拉克 \(\delta\) 函数}的瞬时脉冲流:

\(\Phi_{total}(t) = \underbrace{(1 - \omega(t)) \cdot \Phi_E \cdot \alpha(t)}_{\text{Continuous Elevator Flow}} + \underbrace{\omega(t) \cdot \sum_{k=1}^{n} \delta(t - k \tau) \cdot Q_{surge}}_{\text{Discrete Rocket Pulses}}\)

其中:

\begin{itemize}
\item
  \textbf{\(\tau \approx 28.5\) days}:代表地月会合周期(Synodic
  Month),是物理上的刚性发射窗口。
\item
  \textbf{\(Q_{surge}\)}:单次窗口内的最大并发运输量,由 25
  个发射场共同支撑。
\item
  \textbf{\(\omega(t)\)}:根据系统状态(成本、风险、环境)动态调节的分配权重。
\end{itemize}

\subsubsection{6.1.2 峰值-平均比 (PAR)
与基建冗余辩护}\label{612-ux5cf0ux503c-ux5e73ux5747ux6bd4-par-ux4e0eux57faux5efaux5197ux4f59ux8fa9ux62a4}

引入窗口约束后,系统展现出显著的\textbf{高峰值-平均比(Peak-to-Average
Ratio)}特性。

\pandocbounded{\includegraphics[keepaspectratio,alt={}]{C:/Users/31529/Desktop/MCMworkspace/pictures/pulsed_logistics_trend_fixed.png}}

\textbf{深度洞察与图表分析}:

\begin{itemize}
\item
  \textbf{宏观视图 (Macro-Scale)}:展示了 100
  年间的演化趋势。虽然战略重心从电梯平滑转向火箭与
  ISRU,但这一过程在底层是由数千次脉冲补给组成的。
\item
  \textbf{微观视图
  (Micro-Scale)}:如图所示,太空电梯提供了稳定的``基础负荷''(蓝色区域),而火箭系统表现为强烈的``红色脉冲''。
\item
  \textbf{硬核结论}:在轨道窗口开启的 3-5
  天内,系统瞬时负荷会飙升至平均值的 \textbf{8-10
  倍}。这有力地回击了对``25
  个发射场''是资源浪费的质疑:它们不是为了满足平均需求,而是为了在极短的物理窗口内完成巨大的\textbf{浪涌吞吐
  (Surge Throughput)}。
\end{itemize}

\subsubsection{6.1.3 自适应调节机制:基于惩罚函数的窗口配额优化 (Window
Quota
Optimization)}\label{613-ux81eaux9002ux5e94ux8c03ux8282ux673aux5236ux57faux4e8eux60e9ux7f5aux51fdux6570ux7684ux7a97ux53e3ux914dux989dux4f18ux5316-window-quota-optimization}

为了在脉冲式运输中实现动态平衡,系统建立了一个基于\textbf{前馈-反馈(Feed-forward
\& Feedback)}逻辑的配额算法,通过调节每一个发射窗口的物资配额 \(Q_k\)
来应对潜在的物流中断。

\paragraph{A. 局部优化目标与惩罚函数 (Objective \& Penalty
Function)}\label{a-ux5c40ux90e8ux4f18ux5316ux76eeux6807ux4e0eux60e9ux7f5aux51fdux6570-objective--penalty-function}

系统在每个窗口 \(k\) 开启前,通过求解以下损失函数 \(J_k\)
的极小值来确定发射量:

\[\min_{Q_k} J_k = \underbrace{w_c \cdot \text{Backlog}(k)^2}_{\text{Penalty for Supply Deficit}} + \underbrace{w_e \cdot \exp(Q_k - Q_{env})}_{\text{Penalty for Environmental Surge}}\]

其中:

\begin{itemize}
\item
  \textbf{\(\text{Backlog}(k)\)}:累积的未完成物流赤字。
\item
  \textbf{\(Q_{env}\)}:大气环境自净阈值(Environmental Threshold)。
\item
  \textbf{\(w_c, w_e\)}:动态权重系数。系统根据太空电梯效率
  \(\alpha(t)\) 的实时反馈,在``补给紧迫性''与``环境保护''之间进行博弈。
\end{itemize}

\paragraph{B. 可视化分析:稳态运行 vs. 应急浪涌 (Simulation
Analysis)}\label{b-ux53efux89c6ux5316ux5206ux6790ux7a33ux6001ux8fd0ux884c-vs-ux5e94ux6025ux6d6aux6d8c-simulation-analysis}

\pandocbounded{\includegraphics[keepaspectratio,alt={}]{C:/Users/31529/Desktop/MCMworkspace/pictures/window_quota_optimization.png}}

\textbf{深度洞察与结论}:

\begin{itemize}
\item
  \textbf{稳态平衡 (Windows 1-4)}:如图 6.1.3(a)
  所示,当系统运行平稳时,权重 \(w_e\) 占主导,配额 \(Q_k\) 严格压制在
  \(Q_{env}\)
  以下。此时系统主要依靠太空电梯的低能耗连续流,火箭作为低频补充。
\item
  \textbf{应急浪涌响应 (Windows 5-9)}:模拟在窗口 4 发生的电梯突发故障。

  \begin{itemize}
  \item
    \textbf{环境债的产生}:为了挽救急剧攀升的物流赤字(图
    6.1.3b),系统自动调高 \(w_c\),触发\textbf{``浪涌协议'' (Surge
    Protocol)}。此时,\(Q_k\)
    允许阶跃式突破环境阈值,形成了一段``环境负债期''。
  \item
    \textbf{赤字抹平}:得益于 \textbf{25 个发射场}
    提供的强大带宽并发能力,系统在 5 个窗口内(约 140
    天)即抹平了所有补给赤字,恢复了月球基地的物资安全水位。
  \end{itemize}
\end{itemize}

\paragraph{C. 硬核结论:冗余基建的调峰价值 (Structural
Resilience)}\label{c-ux786cux6838ux7ed3ux8bbaux5197ux4f59ux57faux5efaux7684ux8c03ux5cf0ux4ef7ux503c-structural-resilience}

通过仿真对比发现,若场站数量仅维持在``平均运力''水平(约 10 个),则
\(Q_k\) 受到物理瓶颈限制无法突破
\(Q_{env}\),导致图中的物流赤字(Backlog)将呈线性发散。

这从数学层面论证了 \textbf{``1-2-6 战略''} 的深刻意义:

\begin{enumerate}
\def\labelenumi{\arabic{enumi}.}
\item
  \textbf{物理带宽}:25
  个发射场不仅是为了运输,更是为了提供系统在危急时刻所需的\textbf{瞬时吞吐上限}。
\item
  \textbf{治理智慧}:通过对惩罚函数权重的自适应调节,模型实现了在``保护地球环境''与``保障月球生存''之间的动态决策,展现了系统工程的
\end{enumerate}

\subsection{6.2 硬核反馈:碎片增长与发射频率的非线性耦合 (Debris
Feedback
Loop)}\label{62-ux786cux6838ux53cdux9988ux788eux7247ux589eux957fux4e0eux53d1ux5c04ux9891ux7387ux7684ux975eux7ebfux6027ux8026ux5408-debris-feedback-loop}

为了体现``空间可持续性'',我们建立了一个\textbf{二阶常微分方程反馈回路},描述火箭发射对太空电梯生存环境的负外部性。

\begin{itemize}
\item
  \textbf{碎片增长方程 (Kessler Dynamics)}:
\end{itemize}

\[\frac{dD(t)}{dt} = \underbrace{\sigma \cdot D(t)}_{\text{Self-multiplication}} + \underbrace{\gamma \cdot [N \cdot f(t)]^2}_{\text{Launch Impact}} - \underbrace{\zeta \cdot R_{tech}(t)}_{\text{Active Removal}}\]

其中:

\begin{itemize}
\item
  \(\sigma\) 为凯斯勒效应的自增殖系数。
\item
  \(\gamma \cdot [N \cdot f(t)]^2\)
  项体现了\textbf{非线性反馈}:火箭发射频次越高,产生碰撞碎片的概率呈平方级增长。
\item
  \(\zeta \cdot R_{tech}\) 为碎片清除技术的贡献。
\item
  \textbf{闭环影响}:该方程直接反馈给电梯效率因子
  \(\alpha(t)\)。这种``自残式''反馈机制强制模型在后期削减不必要的发射,转而依赖月球
  ISRU,从而保护电梯的结构安全。

  左图展示了若无制动,碎片密度将突破``电梯安全阈值''(红色虚线);而我们的自适应模型在监测到风险后,会自动触发右图所示的\textbf{``发射制动''
  (Adaptive Launch Throttle)},通过牺牲短期火箭频次,换取太空电梯长达 60
  年的生命安全。

  \pandocbounded{\includegraphics[keepaspectratio,alt={}]{C:/Users/31529/Desktop/MCMworkspace/pictures/debris_feedback_loop.png}}
\end{itemize}

\subsection{6.3 随机性模拟:具有 Scrub Rate 的离散事件 (Stochastic
Process)}\label{63-ux968fux673aux6027ux6a21ux62dfux5177ux6709-scrub-rate-ux7684ux79bbux6563ux4e8bux4ef6-stochastic-process}

我们拒绝使用完美的连续函数,而是将实际运输过程建模为带有\textbf{泊松噪声}的离散事件流:

\begin{itemize}
\item
  \textbf{火箭补给的随机表达}:

  \[\Phi_{R}^{actual}(t) = \sum_{k=1}^{n(t)} P \cdot X_k\]

  其中:

  \begin{itemize}
  \item
    \(n(t)\) 为计划发射次数。
  \item
    \(X_k \sim \text{Bernoulli}(1 - p_{scrub})\) 是一个伯努利随机变量。
  \item
    \(p_{scrub}\) 为发射取消率(Scrub Rate),我们基于 SpaceX
    历史数据设定其遵循正态分布 \(N(0.16, 0.05^2)\)。
  \end{itemize}
\item
  \textbf{结论}:通过 1000 次蒙特卡洛仿真计算出的置信区间显示,即使在
  \(p_{scrub}\) 波动至 25\% 的极端年份,由于我们规划了 \textbf{25
  个发射场} 的盈余运力,系统依然能保持 \(95\%\) 以上的按期交付率。
\end{itemize}

\pandocbounded{\includegraphics[keepaspectratio,alt={}]{C:/Users/31529/Desktop/MCMworkspace/pictures/risk_optimization.png}}

\section{7. 系统韧性:极端失效场景下的压力测试 (System Resilience:
Stress Test under Extreme
Failure)}\label{7-ux7cfbux7edfux97e7ux6027ux6781ux7aefux5931ux6548ux573aux666fux4e0bux7684ux538bux529bux6d4bux8bd5-system-resilience-stress-test-under-extreme-failure}

本章评估系统在面临``黑天鹅''事件时的生存边界。我们通过模拟太空电梯在建设中期的灾难性结构失效,量化``1-2-6
战略''中 25 个发射基地提供的\textbf{动态补偿价值}。

\subsection{7.1 物资赤字的微分动力学模型 (Dynamics of Supply
Depletion)}\label{71-ux7269ux8d44ux8d64ux5b57ux7684ux5faeux5206ux52a8ux529bux5b66ux6a21ux578b-dynamics-of-supply-depletion}

假设在 \(t_{fail}\) 时刻,太空电梯发生彻底断裂。系统的月球物资储备
\(S(t)\) 被建模为一个受限的一阶非齐次微分方程:

\[\frac{dS(t)}{dt} = \Phi_{R}(t) + \Phi_{ISRU}(t) - \Omega(t), \quad \forall t > t_{fail}\]

其中:

\begin{itemize}
\item
  \textbf{\(\Phi_{R}(t)\)}:火箭系统的即时补给流速率。
\item
  \textbf{\(\Phi_{ISRU}(t)\)}:月球原位资源的产出流,随技术成熟度线性或指数增长。
\item
  \textbf{\(\Omega(t)\)}:月球基地的总消耗基准,定义为
  \(\Omega(t) = P(t) \cdot \rho\),其中 \(P(t)\) 为月球人口,\( \rho\)
  为单位人口物资需求。
\end{itemize}

在 \(t_{fail}\) 瞬间,补给速率骤降。由于失效初期
\(\Phi_{R}(t_{fail}) + \Phi_{ISRU}(t_{fail}) < \Omega(t_{fail})\),系统进入\textbf{赤字发散状态},储备量
\(S(t)\) 将迅速跌向生存红线 \(S_{crit}\)。

\subsection{7.2 核心机制:基于单位阶跃函数的``浪涌动员'' (Staggered
Surge
Response)}\label{72-ux6838ux5fc3ux673aux5236ux57faux4e8eux5355ux4f4dux9636ux8dc3ux51fdux6570ux7684ux6d6aux6d8cux52a8ux5458-staggered-surge-response}

为了挽救系统,25
个发射基地必须执行从``稳态''到``超频''的阶跃切换。由于地勤准备和燃料调度的物理限制,我们将这一响应过程建模为具有\textbf{激活延迟
\(\Delta t\)} 的单位阶跃函数(Heaviside Step Function)序列:

\[\Phi_{R}^{surge}(t) = \sum_{i=1}^{N_{pads}} H(t - t_{fail} - \Delta t_i) \cdot \text{Cap}_{i, max}\]

\begin{itemize}
\item
  \textbf{\(H(\cdot)\)}:刻画场站从维护状态转入最大吞吐状态的离散物理跃迁。
\item
  \textbf{\(\Delta t_i\)}:第 \(i\) 个场站的应急响应滞后时间。模型假设
  25 个场站的延迟分布在 5 至 45 天之间,模拟真实的全球动员周期。
\item
  \textbf{\(N_{pads} = 25\)}:这是``1-2-6 战略''的硬核核心。25
  个场站提供的并发带宽是决定系统是否在物资耗尽前反弹的关键变量。
\end{itemize}

\subsection{7.3 生存窗口分析:82 天的``死区''博弈 (Survival Window \&
Pivot
Point)}\label{73-ux751fux5b58ux7a97ux53e3ux5206ux679082-ux5929ux7684ux6b7bux533aux535aux5f08-survival-window--pivot-point}

我们将系统韧性定义为储备速率 \(\frac{dS}{dt}\) 由负转正的临界时刻
\(t^*\),即系统停止``失血''并开始恢复的转折点。

\pandocbounded{\includegraphics[keepaspectratio,alt={}]{C:/Users/31529/Desktop/MCMworkspace/pictures/resilience_hardcore_analysis.png}}

通过数值仿真,我们得出以下硬核结论:

\begin{itemize}
\item
  \textbf{转折点时延 (\(t^* - t_{fail}\))}:在 25
  个场站全功率阶跃动员下,转折点出现在失效后的 \textbf{第 82 天}。如图
  7.1 所示,红色曲线(总供给)在此时刻终于压倒了橙色虚线(消耗率)。
\item
  \textbf{最低存量 (Minimum Inventory)}:物资储备曲线在第 82
  天触及最低点,随后由于火箭系统的``浪涌容量''开始回升。在``1-2-6
  战略''下,最低储备量仍保持在 \(S_{crit}\) 以上 15\%。
\item
  \textbf{对比论证}:若场站数量 \(N\) 仅按照平均运力设定(如 10 个),则
  \(\Phi_{R}^{surge}\) 永远无法覆盖缺口,物资储备将在失效后的第 45
  天耗尽,导致基地结构性崩溃。
\end{itemize}

\subsection{7.4 结论:冗余作为``安全性度量'' (Redundancy as a Safety
Metric)}\label{74-ux7ed3ux8bbaux5197ux4f59ux4f5cux4e3aux5b89ux5168ux6027ux5ea6ux91cf-redundancy-as-a-safety-metric}

本章证明了 \textbf{``1-2-6 战略''}
的韧性本质。系统的安全性不取决于其在``正常状态''下的效率,而取决于其在丧失核心资产后的\textbf{最小生存概率
\(P_{survival}\)}:

\[P_{survival} = P\left( \min_{t} S(t) > S_{crit} \right) > 99.5\%\]

这种通过\textbf{空间冗余(25个发射场)}来对冲\textbf{时间风险(电梯断裂)}的策略,是地外文明长期生存的唯一底层逻辑。

\section{8. 政策建议与基建路径 (Infrastructure \& Policy
Recommendations)}\label{8-ux653fux7b56ux5efaux8baeux4e0eux57faux5efaux8defux5f84-infrastructure--policy-recommendations}

基于前述模型的量化分析,我们为实现 1
亿吨地月转运目标提出以下三维战略框架。该框架不仅关注技术可行性,更强调在极端黑天鹅事件下的生存韧性。

\subsection{8.1 建立``1-2-6''动态物流基准 (Establishing the "1-2-6"
Strategic
Baseline)}\label{81-ux5efaux7acb1-2-6ux52a8ux6001ux7269ux6d41ux57faux51c6-establishing-the-1-2-6-strategic-baseline}

我们建议全球地月物流委员会采纳 \textbf{``1-2-6 战略''}
作为核心基准方案。该方案在数学上已被证明是环境足迹与生存概率之间的帕累托最优解。

\begin{itemize}
\item
  \textbf{长期性 (The 60-Year Commitment)}:模型显示,60 年工期通过分摊
  CAPEX 压力和利用 5-8 代的技术学习曲线,可将总碳排放降低
  \textbf{29.2\%}。任何低于 40
  年的激进方案都将导致地球大气层的环境载荷突破不可逆转的阈值。
\item
  \textbf{冗余即安全 (The 25-Pad
  Mandate)}:政策制定者必须摒弃``效率优先''的短视思维,强制建设
  \textbf{25 个标准化发射基地}。

  \begin{itemize}
  \item
    \textbf{战略辩护}:正如 Section 7.3 所证,这 25
    个场站提供的浪涌容量(Surge Capacity)是电梯失效后月球基地维持 82
    天生存窗口的唯一保障。
  \end{itemize}
\end{itemize}

\subsection{8.2 空间可持续性与``环境债''治理 (Space Sustainability \&
"Environmental
Debt")}\label{82-ux7a7aux95f4ux53efux6301ux7eedux6027ux4e0eux73afux5883ux503aux6cbbux7406-space-sustainability--environmental-debt}

针对 Section 6.2
提出的碎片反馈回路,我们建议实施\textbf{动态发射配额制度 (Dynamic Launch
Quota)}。

\begin{itemize}
\item
  \textbf{自适应配额}:发射频次应与轨道碎片密度 \(D(t)\)
  的监测数据实时挂钩。当碎片增长率偏离基准模型时,应强制削减火箭发射,转而通过提高月球原位资源利用
  (ISRU) 的比率来对冲运力缺口。
\item
  \textbf{环境偿还机制}:在发生类似 Section 7
  的应急浪涌补给时,允许产生短期的``环境债'',但必须在随后的稳定运行期内通过碳补偿或碎片清理技术(Active
  Debris Removal)进行对冲偿还。
\end{itemize}

\subsection{8.3 跨系统协同优化方案 (Cross-System Collaborative
Optimization)}\label{83-ux8de8ux7cfbux7edfux534fux540cux4f18ux5316ux65b9ux6848-cross-system-collaborative-optimization}

地月物流不应被视为单一工具的运行,而应作为一套\textbf{分布式韧性网络}进行管理。

\begin{itemize}
\item
  \textbf{混合运输范式}:在 2050-2070 年的建设期,应优先利用太空电梯承担
  80\% 的大宗散装物资(如结构钢材、燃料);在 2070
  年后的运营期,逐步增加火箭舰队的比例,以利用其极高的调度灵活性应对脉冲式需求。
\item
  \textbf{标准化接口}:为了确保 25 个基地能像 Section 7.2
  描述的那样迅速切换至``全功率阶跃模式'',所有基地的燃料补给、有效载荷封装必须实现全球标准化,消除黑天鹅事件下的物流瓶颈。
\end{itemize}

\subsection{8.4 结论:迈向星际文明的稳健步伐 (Conclusion: A Robust Step
Toward Multi-Planetary
Civilization)}\label{84-ux7ed3ux8bbaux8fc8ux5411ux661fux9645ux6587ux660eux7684ux7a33ux5065ux6b65ux4f10-conclusion-a-robust-step-toward-multi-planetary-civilization}

通过本研究的多目标优化与韧性压力测试,我们证明了地月物流系统的成功不仅取决于最先进的推进技术,更取决于\textbf{合理的结构化冗余}。1
亿吨的任务量虽然宏大,但在 60 年的自适应管理下,辅以 25
个发射基地的战略保障,人类完全可以在保护母星环境的前提下,建立起跨越 38
万公里的永恒生命线。

\end{document}
